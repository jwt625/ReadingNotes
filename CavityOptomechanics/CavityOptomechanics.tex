%!TEX program = xelatex
%!TEX builder = latexmk
% or can be latexmk texify
%!TEX option = 
% -shell-escape -8bit % For minted package
%!TEX root = ...
\documentclass[8pt,a4paper,twocolumn]{article} % titlepage表示标题单独页
\usepackage[noindent]{ctex} % ctex套用英文标题格式 (建议在英文论文混排中文时使用) ,
% ctexcap套用中文格式 (等同于\documentclass{ctexart}) 
% \renewcommand{\figurename}{图}
% \renewcommand{\tablename}{表}
% \renewcommand{\contentsname}{目录}
% \renewcommand\refname{参考文献}
% \renewcommand{\thefigure}{\chinese{figure}} % 将图片计数改为汉字数字
% \renewcommand{\thetable}{\chinese{table}} % 将表格计数改为汉字数字
\usepackage[top=0.75in,bottom=0.75in,left=0.75in,right=0.75in]{geometry} % 页边距设置
% \usepackage{multicol}页面内多行包
\usepackage[CJKbookmarks]{hyperref} % 给pdf文档添加互动式链接和书签
\hypersetup{linktocpage=true}%make the link of the content on the number of page
% \userpackage{wrapfig} % 图文绕排
% \usepackage{xeCJK} % to get my Chinese name
% \setCJKmainfont{SimSun}
% \usepackage[parfill]{parskip} % 增加段间行距
\usepackage{amsmath,amssymb,esint} % 数学公式类宏包;最末为积分符号拓展
\allowdisplaybreaks[0]% 允许多行公式间换页, 用//*表示不允许换页
\numberwithin{equation}{section} % 公式编号包含章节
\usepackage{bm} % 加粗 (用于vector) 
\usepackage{mathrsfs} % mathscr font
% \usepackage{textcomp} % 符号包, 不能用于数学模式, 建议不要和SIunits混用
% \usepackage[squaren]{SIunits} % 科学单位包, 可以用于数学模式
% (为了统一不要和textcomp混用) , squaren选项消除和amssymb的冲突
\usepackage{siunitx} % 淘汰掉上面这个宏包吧, 现在用的是
% \num{123}, \si{kg.m/s^2}, 
% \si{\electronvolt\per\square\clight}, \SI{123}{\micro\metre}
\usepackage{extarrows} % 长箭头, 长等号etc.
\usepackage{graphicx} % 插图宏包
% \usepackage{picinpar} % 图文绕排
\usepackage{array} % 表格宏包
% \usepackage{longtable} % 长表格宏包
\usepackage{multirow} % 多行合并的表格宏包
% \usepackage{booktabs} % 表格线宏包
\usepackage{braket} % 狄拉克符号

% \usepackage[basic,box,gate,oldgate,ic,optics,physics]{circ} % 电路图宏包
% \usepackage[normalem]{ulem} % 下划线, 删除线等宏包, 参数表示不修改\emph{}格式
% \usepackage{mychemistry} % 化学宏包, 包含mhchem和chemfig
% \usepackage[version=3]{mhchem} % 化学宏包, 包含mhchem和chemfig
% \usepackage[symbol]{footmisc} % 脚注拓展, 选项表示用符号做脚注记号
% \usepackage{listings} % 代码段宏包
% \lstset{numbers=left,frame=shadowbox,%
% basicstyle=\ttfamily, commentstyle=\fontseries{lc}\selectfont\itshape, %
% columns=fullflexible, breaklines=true, escapeinside={(*@}{@*)}}
% \usepackage{minted} % 具有 Python 支持的代码宏包

% \renewcommand*{\vec}[1]{\bm{#1}} % 矢量的格式, 这里是加粗
\newcommand{\dif}{\,\mathrm d}
\newcommand\mi{\mathrm{i}}
\newcommand\e{\mathrm{e}} % 定义数学模式中常用的正体字符
\newcommand\Y{\mathrm{Y}}
\newcommand\cc{\mathrm{c.c.}}
\DeclareMathOperator{\Imag}{Im}
\bibliographystyle{unsrt}

\begin{document}\small
	\title{Cavity Optomechanics}
	\author{Wentao Jiang}
	\date{}
	\maketitle
	\tableofcontents

	\section{Introduction} % (fold)
	\label{sec:introduction}
		Historical review:
		\begin{itemize}
    \setlength{\itemsep}{0.5pt}
    \setlength{\parsep}{0.5pt}
    \setlength{\parskip}{0.5pt}
			\item Ashkin, focused laser beams can trap and control dielectric particles; Laser cooling; \\
			Application: optical atomic clocks, precision measurements
			\item Braginsky, dynamical influence of radiation pressure; quantum fluctuations of it, established the standard quantum limit for continuous position detection
			\item theoretical: squeezing of light, QND detection of the light intensity, quantum nonlinearities for extremely strong optomechanical coupling, give rise to nonclassical and entangled states of the light field and the mechanics
			\item experimental: optical feedback cooling; feedback damping, self-induced oscillations
			\item systems: membranes; nanorods; microdisks; microspheres; optical waveguides; nanomechanical beam inside a superconducting transmission line microwave cavity
			\item motivations: sensitive optical detection of small forces, displacements, masses and accelerations; interconvert information between solid-state qubits and flying photonic qubits
		\end{itemize}
	% section introduction (end)

	\section{Optical Cavities and Mechanical Resonators} % (fold)
	\label{sec:optical_cavities_and_mechanical_resonators}
		\subsection{Optical resonators} % (fold)
		\label{sub:optical_resonators}
			F-P resonator (etalon):
			\begin{itemize}
				\item angular frequency: $ \omega_{\text{cav},m}\approx m\cdot \pi(c/L) $
				\item free spectral range (FSR): $ \Delta \omega_{\text{FSR}}=\pi \frac{c}{L} $
				\item optical finesse: $\mathcal{F}\equiv\Delta \omega_{\text{FSR}}/ \kappa$
				\item quality factor: $Q_{\text{opt}}=\omega_{\text{cav}} \tau $
				\item total cavity loss rate: $ \kappa=\kappa_{\text{ex}}+\kappa_0 $\\
				where $\kappa_{\text{ex}} $ refers to the input coupling loss rate\\
				photons going into the $\kappa_0$ decay channel won't be recorded
			\end{itemize}
		% subsection optical_resonators (end)

		\subsection{Input-output formalism} % (fold)
		\label{sub:input_output_formalism}
			
			Master equations: only internal dynamics is of interest\\
			Input-output theory: include the light field being emitted by the cavity, formulated on the level of Heisenberg equations of motion, describing the time evolution of the field amplitude $\hat{a}$ inside the cavity:
			\begin{equation}
			\label{eqt:fieldApmInCavity}
				\dot{\hat{a}}=-\frac{\kappa}{2}\hat{a}+i \Delta \hat{a} +\sqrt{\kappa_{\text{ex}}  }\hat{a}_{\text{in}}+ \sqrt{\kappa_0}\hat{f}_{\text{in}} 
			\end{equation}
			where laser detuning $\Delta=\omega_L- \omega_{\text{cav}} $, $\hat{a}_{\text{in}}$ should be regarded as a stochastic quantum field. The field is normalized that
			\begin{equation}
				P=\hbar \omega_L \left< \hat{a}_{\text{in}}^{\dagger} \hat{a}_{\text{in}} \right>
			\end{equation}
			is the input power launched into the cavity. The same kind of description holds for the ``unwanted'' channel $\hat{f}_{\text{in}} $.

			The field reflected from the F-P resonator is given by
			\begin{equation}
			\label{eqt:fieldReflectedFromFP}
				\hat{a}_{\text{out}}=\hat{a}_{\text{in}}-\sqrt{\kappa_{\text{ex}}} \hat{a}
			\end{equation}

			After taking average of eqt \ref{eqt:fieldApmInCavity}, \ref{eqt:fieldReflectedFromFP} and $ \left<\hat{f}_{\text{in}} \right> =0$:
			\begin{gather}
				\left<\hat{a} \right>= \frac{ \sqrt{\kappa_{\text{ex}}} \left< \hat{a}_{\text{in}} \right>}{\kappa/2-i \Delta}\\
				\chi_{\text{opt}}(\omega)\equiv \frac{1}{-i(\omega+\Delta)+\kappa/2 }
			\end{gather}
			This is adequate as long as $ \kappa \ll \Delta \omega_{\text{FSR}} $, i.e., $Q\gg1$

			The steady state cavity population
			\begin{equation}
				\bar{n}_{\text{cav}}=\left| \left< \hat{a} \right> \right|^2 = \frac{\kappa_{\text{ex}}}{\Delta^2+(\kappa/2)^2} \frac{P}{\hbar \omega_L}
			\end{equation}

			Reflection amplitude:
			\begin{equation}
				\mathcal{R}=\frac{\left< \hat{a}_{\text{out}} \right>}{\left< \hat{a}_{\text{in}} \right>}=\frac{(\kappa_0- \kappa_{\text{ex}}  )/2-i \Delta}{(\kappa_0+ \kappa_{\text{ex}}  )/2-i \Delta}
			\end{equation}

			If the external coupling dominates the cavity losses ($ \kappa_{\text{ex}}\approx \kappa\gg \kappa_0 $), the cavity is called ``\textbf{overcoupled}'' and $ \left|\mathcal{R}\right|^2 \approx 1$. The case where $ \kappa_{\text{ex}}=\kappa_0  $ is called ``\textbf{critical coupling}'', where $\mathcal{R}(\Delta=0)=0 $ on resonance. This implies that the input power is either fully dissipated within the resonator or fully transmitted. The situation $ \kappa_{\text{ex}} \ll \kappa_0  $ is referred to as ``\textbf{undercoupling}'' and is associated with cavity losses dominated by intrinsic losses, leading to an effective loss of information.


		% subsection input_output_formalism (end)

		\subsection{Mechanical resonators} % (fold)
		\label{sub:mechanical_resonators}
			\begin{itemize}
				\item displacement field: $\vec{u}_n(\vec r) $, $n$ denotes the mode
				\item vibration frequency: $\Omega_m$
				\item energy damping rate: $ \Gamma_m $
				\item mechanical quality factor: $ Q_m = 1/\delta \Phi = \Omega_m/\Gamma_m $
				\item global amplitude: $x (\vec r) $
				\item a suitable mode function: $ \vec u (\vec r,t)=x(t)\cdot \vec u(\vec r) $
			\end{itemize}
			then the temperal evolution of $x(t) $:
			\begin{equation}
			\label{eqt:eqtOfXforMechResonator}
				m_{\text{eff}} \frac{d^2 x(t)}{dt^2} +m_{\text{eff}} \frac{dx(t)}{dt}+m_{\text{eff}} \Omega_m^2 x(t)=F_{\text{ex}}(t)
			\end{equation}
			$F_{\text{ex}}(t)$ denotes the sum of all forces acting on the mechanical oscillator. In the absence of external forces, it is given by the Langevin force.

			Eqt \ref{eqt:eqtOfXforMechResonator} can be solved easily in frequency space:
			\begin{gather}
				x(\omega)=\int dt e^{i \omega t} x(t)\\
				x_m(\omega)=\left[ m_{\text{eff}} \left( \Omega_m^2-w	^2 \right)-im_{\text{eff}} \Gamma_m \omega \right]^{-1}
			\end{gather}

			The quantum mechanical treatment:
			\begin{gather}
				\hat{H}=\hbar \Omega_m \hat{b}^{\dagger}\hat{b}+\frac{1}{2} \hbar \Omega_m\\
				\hat{x}=x_{\text{XPF}}(\hat{b}+\hat{b}^{\dagger})\\
				\hat{p}= -i m_{\text{eff}} \Omega_m x_{\text{XPF}}(\hat{b}-\hat{b}^{\dagger})
			\end{gather}
			where
			\begin{equation}
				x_{\text{ZPF}}=\sqrt{ \frac{\hbar}{2 m_{\text{eff}} \Omega_m } }
			\end{equation}
			is the zero-point fluctuation amplitude.

			Coupled to a high-temperature bath:
				\begin{gather}
					\frac{d}{dt}\bar n = -\Gamma_m (\bar n-\bar n_{\text{th}})\\
					\frac{d}{dt}\bar n(t=0) = \Gamma_m \bar n_{\text{th}}\approx \frac{k_B T_{\text{bath}} }{\hbar Q_m}
				\end{gather}

			\subsubsection{Mechanical dissipation} % (fold)
			\label{ssub:mechanical_dissipation}
				
				\begin{itemize}
					\item viscous damping
					\item clamping losses
					\item fundamental anharmonic effects
					\item materials-induced losses
				\end{itemize}
			% subsubsection mechanical_dissipation (end)

			\subsubsection{Susceptibility, noise spectra, fluctuation-dissipation theorem} % (fold)
			\label{ssub:susceptibility_noise_spectra_fluctuation_dissipation_theorem}
				Gated Fourier transform:
				\begin{equation}
					\tilde x(\omega)=\frac{1}{\sqrt \tau} \int_0^ \tau x(t)e^{i \omega t }dt
				\end{equation}

				Noise power spectral density:
				\begin{equation}
					S_{xx}\equiv \int_{-\infty}^{+\infty} \left< x(t)x(0) \right> e^{i \omega t}dt
				\end{equation}

				Wiener-Khinchin theorem:
				\begin{equation}
					\lim_{\tau\rightarrow \infty} \left<\left|\tilde x(\omega) \right|^2\right>=S_{xx}(\omega)
				\end{equation}

				The experimentally measured mechanical noise spectrum yields the variance of the mechanical displacement $\left< x^2\right>$:
				\begin{equation}
					\int_{-\infty}^{+\infty} S_{xx}(\omega)\frac{d \omega}{2 \pi}=\left< x^2\right>
				\end{equation}

				The fluctuation-dissipation theorem (FDT):
				\begin{equation}
					S_{xx}(\omega)=2 \frac{k_B T}{\omega} \text{Im} \chi_m (\omega)
				\end{equation}
				where $\chi_m (\omega)$ denotes the mechanical susceptibility.

				The quantum FDT:
				\begin{equation}
					S_{xx}(\omega)= \frac{\hbar}{1-\exp (-\hbar \omega/k_B T) } \text{Im} \chi_{xx} (\omega)
				\end{equation}
			% subsubsection susceptibility_noise_spectra_fluctuation_dissipation_theorem (end)

		% subsection mechanical_resonators (end)


	% section optical_cavities_and_mechanical_resonators (end)

	\section{Principles of Optomechanical Coupling} % (fold)
	\label{sec:principles_of_optomechanical_coupling}
	
	% section principles_of_optomechanical_coupling (end)

	%\bibliography{CQED}
\end{document}