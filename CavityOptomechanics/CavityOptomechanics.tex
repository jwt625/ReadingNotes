%!TEX program = xelatex
%!TEX builder = latexmk
% or can be latexmk texify
%!TEX option = 
% -shell-escape -8bit % For minted package
%!TEX root = ...
\documentclass[8pt,a4paper,twocolumn]{article} % titlepage表示标题单独页
\usepackage[noindent]{ctex} % ctex套用英文标题格式 (建议在英文论文混排中文时使用) ,
% ctexcap套用中文格式 (等同于\documentclass{ctexart}) 
% \renewcommand{\figurename}{图}
% \renewcommand{\tablename}{表}
% \renewcommand{\contentsname}{目录}
% \renewcommand\refname{参考文献}
% \renewcommand{\thefigure}{\chinese{figure}} % 将图片计数改为汉字数字
% \renewcommand{\thetable}{\chinese{table}} % 将表格计数改为汉字数字
\usepackage[top=0.75in,bottom=0.75in,left=0.75in,right=0.75in]{geometry} % 页边距设置
% \usepackage{multicol}页面内多行包
\usepackage[CJKbookmarks]{hyperref} % 给pdf文档添加互动式链接和书签
\hypersetup{linktocpage=true}%make the link of the content on the number of page
% \userpackage{wrapfig} % 图文绕排
% \usepackage{xeCJK} % to get my Chinese name
% \setCJKmainfont{SimSun}
% \usepackage[parfill]{parskip} % 增加段间行距
\usepackage{amsmath,amssymb,esint} % 数学公式类宏包;最末为积分符号拓展
\allowdisplaybreaks[0]% 允许多行公式间换页, 用//*表示不允许换页
\numberwithin{equation}{section} % 公式编号包含章节
\usepackage{bm} % 加粗 (用于vector) 
\usepackage{mathrsfs} % mathscr font
% \usepackage{textcomp} % 符号包, 不能用于数学模式, 建议不要和SIunits混用
% \usepackage[squaren]{SIunits} % 科学单位包, 可以用于数学模式
% (为了统一不要和textcomp混用) , squaren选项消除和amssymb的冲突
\usepackage{siunitx} % 淘汰掉上面这个宏包吧, 现在用的是
% \num{123}, \si{kg.m/s^2}, 
% \si{\electronvolt\per\square\clight}, \SI{123}{\micro\metre}
\usepackage{extarrows} % 长箭头, 长等号etc.
\usepackage{graphicx} % 插图宏包
% \usepackage{picinpar} % 图文绕排
\usepackage{array} % 表格宏包
% \usepackage{longtable} % 长表格宏包
\usepackage{multirow} % 多行合并的表格宏包
% \usepackage{booktabs} % 表格线宏包
\usepackage{braket} % 狄拉克符号

% \usepackage[basic,box,gate,oldgate,ic,optics,physics]{circ} % 电路图宏包
% \usepackage[normalem]{ulem} % 下划线, 删除线等宏包, 参数表示不修改\emph{}格式
% \usepackage{mychemistry} % 化学宏包, 包含mhchem和chemfig
% \usepackage[version=3]{mhchem} % 化学宏包, 包含mhchem和chemfig
% \usepackage[symbol]{footmisc} % 脚注拓展, 选项表示用符号做脚注记号
% \usepackage{listings} % 代码段宏包
% \lstset{numbers=left,frame=shadowbox,%
% basicstyle=\ttfamily, commentstyle=\fontseries{lc}\selectfont\itshape, %
% columns=fullflexible, breaklines=true, escapeinside={(*@}{@*)}}
% \usepackage{minted} % 具有 Python 支持的代码宏包

% \renewcommand*{\vec}[1]{\bm{#1}} % 矢量的格式, 这里是加粗
\newcommand{\dif}{\,\mathrm d}
\newcommand\mi{\mathrm{i}}
\newcommand\e{\mathrm{e}} % 定义数学模式中常用的正体字符
\newcommand\Y{\mathrm{Y}}
\newcommand\cc{\mathrm{c.c.}}
\DeclareMathOperator{\Imag}{Im}
\bibliographystyle{unsrt}

\begin{document}\small
	\title{Cavity Optomechanics}
	\author{Wentao Jiang}
	\date{}
	\maketitle
	\tableofcontents

	\section{Introduction} % (fold)
	\label{sec:introduction}
		Historical review:
		\begin{itemize}
    \setlength{\itemsep}{0.5pt}
    \setlength{\parsep}{0.5pt}
    \setlength{\parskip}{0.5pt}
			\item Ashkin, focused laser beams can trap and control dielectric particles; Laser cooling; \\
			Application: optical atomic clocks, precision measurements
			\item Braginsky, dynamical influence of radiation pressure; quantum fluctuations of it, established the standard quantum limit for continuous position detection
			\item theoretical: squeezing of light, QND detection of the light intensity, quantum nonlinearities for extremely strong optomechanical coupling, give rise to nonclassical and entangled states of the light field and the mechanics
			\item experimental: optical feedback cooling; feedback damping, self-induced oscillations
			\item systems: membranes; nanorods; microdisks; microspheres; optical waveguides; nanomechanical beam inside a superconducting transmission line microwave cavity
			\item motivations: sensitive optical detection of small forces, displacements, masses and accelerations; interconvert information between solid-state qubits and flying photonic qubits
		\end{itemize}
	% section introduction (end)

	\section{Optical Cavities and Mechanical Resonators} % (fold)
	\label{sec:optical_cavities_and_mechanical_resonators}
		
	% section optical_cavities_and_mechanical_resonators (end)

	%\bibliography{CQED}
\end{document}