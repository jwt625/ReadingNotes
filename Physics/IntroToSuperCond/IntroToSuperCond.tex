%!TEX program = xelatex
%!TEX builder = latexmk
% or can be latexmk texify
%!TEX option = 
% -shell-escape -8bit % For minted package
%!TEX root = ...
\documentclass[8pt,a4paper,twocolumn]{article} % titlepage表示标题单独页
\usepackage[noindent]{ctex} % ctex套用英文标题格式 (建议在英文论文混排中文时使用) ,
% ctexcap套用中文格式 (等同于\documentclass{ctexart}) 
% \renewcommand{\figurename}{图}
% \renewcommand{\tablename}{表}
% \renewcommand{\contentsname}{目录}
% \renewcommand\refname{参考文献}
% \renewcommand{\thefigure}{\chinese{figure}} % 将图片计数改为汉字数字
% \renewcommand{\thetable}{\chinese{table}} % 将表格计数改为汉字数字
\usepackage[top=0.75in,bottom=0.75in,left=0.75in,right=0.75in]{geometry} % 页边距设置
% \usepackage{multicol}页面内多行包
\usepackage[CJKbookmarks]{hyperref} % 给pdf文档添加互动式链接和书签
\hypersetup{linktocpage=true}%make the link of the content on the number of page
% \userpackage{wrapfig} % 图文绕排
% \usepackage{xeCJK} % to get my Chinese name
% \setCJKmainfont{SimSun}
% \usepackage[parfill]{parskip} % 增加段间行距
\usepackage{amsmath,amssymb,esint} % 数学公式类宏包;最末为积分符号拓展
\allowdisplaybreaks[0]% 允许多行公式间换页, 用//*表示不允许换页
\numberwithin{equation}{section} % 公式编号包含章节
\usepackage{bm} % 加粗 (用于vector) 
\usepackage{mathrsfs} % mathscr font
% \usepackage{textcomp} % 符号包, 不能用于数学模式, 建议不要和SIunits混用
% \usepackage[squaren]{SIunits} % 科学单位包, 可以用于数学模式
% (为了统一不要和textcomp混用) , squaren选项消除和amssymb的冲突
\usepackage{siunitx} % 淘汰掉上面这个宏包吧, 现在用的是
% \num{123}, \si{kg.m/s^2}, 
% \si{\electronvolt\per\square\clight}, \SI{123}{\micro\metre}
\usepackage{extarrows} % 长箭头, 长等号etc.
\usepackage{graphicx} % 插图宏包
% \usepackage{picinpar} % 图文绕排
\usepackage{array} % 表格宏包
% \usepackage{longtable} % 长表格宏包
\usepackage{multirow} % 多行合并的表格宏包
% \usepackage{booktabs} % 表格线宏包
\usepackage{braket} % 狄拉克符号

% \usepackage[basic,box,gate,oldgate,ic,optics,physics]{circ} % 电路图宏包
% \usepackage[normalem]{ulem} % 下划线, 删除线等宏包, 参数表示不修改\emph{}格式
% \usepackage{mychemistry} % 化学宏包, 包含mhchem和chemfig
% \usepackage[version=3]{mhchem} % 化学宏包, 包含mhchem和chemfig
% \usepackage[symbol]{footmisc} % 脚注拓展, 选项表示用符号做脚注记号
% \usepackage{listings} % 代码段宏包
% \lstset{numbers=left,frame=shadowbox,%
% basicstyle=\ttfamily, commentstyle=\fontseries{lc}\selectfont\itshape, %
% columns=fullflexible, breaklines=true, escapeinside={(*@}{@*)}}
% \usepackage{minted} % 具有 Python 支持的代码宏包

% \renewcommand*{\vec}[1]{\bm{#1}} % 矢量的格式, 这里是加粗
\newcommand{\dif}{\,\mathrm d}
\newcommand\mi{\mathrm{i}}
\newcommand\e{\mathrm{e}} % 定义数学模式中常用的正体字符
\newcommand\Y{\mathrm{Y}}
\newcommand\cc{\mathrm{c.c.}}
\DeclareMathOperator{\Imag}{Im}
\bibliographystyle{unsrt}

\begin{document}\small
	\title{Introduction to Superconductivity}
	\author{Wentao Jiang}
	\date{}
	\maketitle
	\tableofcontents

	\section{Historical overview} % (fold)
	\label{sec:historical_overview}
		\subsection{The Basic Phenomena} % (fold)
		\label{sub:the_basic_phenomena}
			\begin{itemize}
				\item perfect conductivity
				\item perfect diamagnetism, $H_c(T) \approx H_c(0)\left[ 1-(T/T_c)^2 \right]$
			\end{itemize}

		% subsection the_basic_phenomena (end)

		\subsection{The London Equations} % (fold)
		\label{sub:the_london_equations}
			\begin{align}
				&\bm E = \frac{\partial}{\partial t}( \Lambda \bm J_s )\\
				&\bm h = -c \nabla\times (\Lambda \bm J_s)\\
				&\Lambda = \frac{4 \pi \lambda^2}{c^2}=\frac{m}{n_s e^2}
			\end{align}

			$\bm h$: microscopic flux density

			$\bm B$: macroscopic average value

			$n_s$: number density of superconducting electrons

			Combining Maxwell equation $ \nabla\times \bm h=4 \pi \bm J/c$ leads to $\nabla^2 \bm h = h/\lambda^2$, predicting penetration depth $ \lambda $. Emprically:
			\begin{equation}
				\lambda(T)\approx \lambda(0)\left[ 1-(T/T_c)^4 \right]^{-1/2}
			\end{equation}

			Consider a perfect normal conductor in a uniform $\bm E$ field, $d(m\bm v)/dt = e\bm E,\bm J=ne\bm v$. But it's not rigorous for spacially nonuniform fields within $\lambda$, for which the LEs are most useful. It's because the response of an electron gas to $\bm E$ field is nonlocal.

			A more profound motivation for LEs is a quantum one by considering $\bm p = (m \bm v+e \bm A/c)$ to have zero expectancy for ground state, i.e.,
			\begin{align}
				&\left< \bm v_s \right>=\frac{-e \bm A}{mc}\\
				&\bm J_s = n_s e\left< \bm v_s \right>=\frac{-\bm A}{\Lambda c}
			\end{align}
			which contains the two LEs in a compact form (it's not gauge-invariant, London gauge $ \nabla\cdot\bm A=0 $ is required).
		% subsection the_london_equations (end)

		\subsection{The Pippard Nonlocal Electrodynamics} % (fold)
		\label{sub:the_pipard_nonlocal_electrodynamics}
			Introduce the coherence length $ \xi $ to propose a nonlocal generalization of the LEs, in analogy to Chamber's nonlocal generalization of Ohm's law:
			\begin{equation*}
				\bm J(\bm r)=\frac{3 \sigma}{4 \pi l}\int \frac{\bm R[\bm R\cdot \bm E (\bm r')] e^{-R/l}}{R^4}d\bm r'
			\end{equation*}

			From uncertainty principle, for electrons play a major role in superconductive phenomenon, $E\sim kT_c, \Delta p\approx kT_c/v_F$, thus
			\begin{equation}
				\xi_0 = a \frac{\hbar v_F}{kT_c}
			\end{equation}
			hence
			\begin{equation}
				\bm J_s(\bm r)=-\frac{3 }{4 \pi \xi_0 \Lambda c}\int \frac{\bm R[\bm R\cdot \bm A (\bm r')] e^{-R/\xi}}{R^4}d\bm r'
			\end{equation}
			instead of $ \bm J_s = n_s e\left< \bm v_s \right>=\frac{-\bm A}{\Lambda c} $, where $ 1/\xi = 1/\xi_0 + 1/l $.
		% subsection the_pipard_nonlocal_electrodynamics (end)

		\subsection{The Energy Gap and the BCS Theory} % (fold)
		\label{sub:the_energy_gap_and_the_bcs_theory}
			Experimental hints:

			Electronic specific heat: $C_{es}\approx \gamma T_c a e^{-bT_c/T} $, while normal state $C_{en}=\gamma T $

			Spectroscopic measurement gives minimum energy $E_g$ to create excitations, while thermal one measures $E_g/2$ per statistically independent particle, suggesting pair production.

			Key prediction of BCS: $E_g(0) = 2 \Delta(0) =3.528kT_c $ for $T\ll T_c$.

		% subsection the_energy_gap_and_the_bcs_theory (end)

		\subsection{The Ginzburg-Landau Theory} % (fold)
		\label{sub:the_ginzburg_landau_theory}
			7 years before BCS, they introduce a complex pseudowavefunction as an order parameter within Landau's general theory of 2nd order phase transitions. Eqt for $\psi$ are obtained from a variational principle and expansion of the free energy in powers of $\psi$ and $\nabla \psi$ with coefficients $\alpha$ and $\beta$:
			\begin{align}
				&n_s = \left| \psi(x) \right|^2\\
				&\frac{1}{2m^*}\left( \frac{\hbar}{i}\nabla -\frac{e^*}{c}A \right)^2 \psi+\beta|\psi|^2 \psi=-\alpha(T) \psi\\
				&\bm J_s =\frac{e^* \hbar}{i2m^*}(\psi^* \nabla \psi -\psi \nabla \psi^*)-\frac{{e^*}^2}{m^*c}|\psi|^2\bm A
			\end{align}

			GL coherence length: $\xi(T)=\frac{\hbar}{|2m^* \alpha(T) |^{1/2}} $

			GL parameter: $ \kappa = \frac{\lambda}{\xi} $

		% subsection the_ginzburg_landau_theory (end)

		\subsection{Type II Superconductors} % (fold)
		\label{sub:type_ii_superconductors}
			$\xi<\lambda$ leads to a negative surface energy, so that subdivision proceeds until limited by the microscopic length $\xi$.

			Another result: mixed state, or Schubnikov phase. Between $H_{c1}$ and $ H_{c2} $, the flux penetrates in a regular array of flux tubes carrying flux of
			\begin{equation}
				\Phi_0=\frac{hc}{2e}=2.07\times10^{-7}G/cm^2
			\end{equation}
		% subsection type_ii_superconductors (end)

		\subsection{Josephson Tunneling and Flux Quantization} % (fold)
		\label{sub:josephson_tunneling_and_flux_quantization}
			Phase and particle number are conjugate variables:
			\begin{equation}
				\Delta N \Delta \phi \gtrsim 1
			\end{equation}

			Josephson predicted pairs should be able to tunnel between two superconductor even at zero voltage difference:
			\begin{equation}
				J=J_c \sin (\phi_1- \phi_2)
			\end{equation}
			and with a voltage difference $V_{12}$, the phase difference $\phi = 2eV_{12}t/\hbar $ so that the current would oscillate.

			Single-valuedness of $\psi = |\psi| e^{i \phi} $ requires that the fluxoid introduced by F. London:
			\begin{equation}
				\Phi'=\Phi+\frac{m^* c}{e^*} \oint \frac{\bm J_s \cdot d \bm s}{|\psi|^2}
			\end{equation}
			to take only integral multiples of $\Phi_0 = hc/2e $, where $\Phi = \oint \bm A\cdot d\bm s$ is the ordinary magnetic flux. For ring with thickness comparable to $\lambda$, $J_s=0$ on the integration path, $\Phi=\Phi'=n \Phi_0$.
		% subsection josephson_tunneling_and_flux_quantization (end)

		\subsection{Fluctuation and Nonequilibrium Effects} % (fold)
		\label{sub:fluctuation_and_nonequilibrium_effects}
			
		% subsection fluctuation_and_nonequilibrium_effects (end)

		\subsection{High-temperature Superconductivity} % (fold)
		\label{sub:high_temperature_superconductivity}
		
		% subsection high_temperature_superconductivity (end)

	% section historical_overview (end)

	\section{Introduction to Electrodynamics of Superconductivity} % (fold)
	\label{sec:intro_to_electrodynamics_of_superconductivity}
		\subsection{Screening} % (fold)
		\label{sub:screening}
			From the LEs:
			\begin{equation}
				d\bm J_s/dt = (c^2/4 \pi \lambda^2	)\bm E
			\end{equation}
			Taking the time derivative of the Maxwell eqt $ \nabla \times \bm h=4 \pi \bm J/c $ and eliminating $\partial \bm h/\partial t $ using $ \nabla \times \bm E=-(1/c) \partial \bm h/\partial t $, we obtain
			\begin{equation}
				-\nabla \times \nabla\times\bm E=\nabla^2\bm E = \bm E/\lambda^2
			\end{equation}

			Taking the curl of the Maxwell eqt $ \nabla\times\bm h=(4 \pi/c)\bm J $ and by substituting $-\bm h/c \Gamma = \nabla\times\bm J$, we see that
			\begin{equation}
				\nabla^2\bm h=(1/\lambda^2)\bm h
			\end{equation}
			where $ \lambda^2=mc^2/4 \pi n_s e^2 $

			\subsubsection{E.G. Flat Slab in Parallel Magnetic Field} % (fold)
			\label{subs:e_g_flat_slab_in_parallel_magnetic_field}
				Boundary: $h=H_a$ at the two surfaces at $x=\pm d/2$\\
				Solution:
				\begin{equation}
					h=H_a \frac{\cosh (x/\lambda)}{\cosh(d/2 \lambda	)}
				\end{equation}

				Average over the sample thickness $d$, gives
				\begin{equation}
				B\equiv \bar h \equiv H_a +4 \pi M = H_a \frac{2 \lambda}{d}\tanh \frac{d}{2 \lambda}
				\end{equation}
				from which when $d\gg \lambda, B\rightarrow 0$ and $M\rightarrow -H_a/4 \pi $. This is \textbf{Meissner effect} limit of perfect diamagnetism of bulk superconductors. On the other hand, when $d\ll \lambda$, series expansion shows
				\begin{equation}
					M\rightarrow -(H_a/4 \pi)(d^2/12 \lambda^2)
				\end{equation}


			% subsubsection e_g_flat_slab_in_parallel_magnetic_field (end)

			\subsubsection{Critical Current of Wire} % (fold)
			\label{ssub:critical_current_of_wire}
				Consider a long superconducting wire of radius $a\gg \lambda$, carrying a current $I$. The current generate self-field at the surface of the wire of magnitude $H=2I/ca \le H_c$, hence $I_c =caH_c/2 $, which scales with the perimeter, suggesting that the current flows only in a surface layer of constant thickness. It can be confirmed analytically by solving the LEs and MEs in this geometry that the thickness of the layer is $\lambda$, hence the critical current density
				\begin{equation}
					J_c=\frac{c}{4 \pi} \frac{H_c}{\lambda}
				\end{equation}
				This $J_c$ also holds for wires thinner than $\lambda$, where the current density is nearly uniform and $I_c$ is proportional to the cross-sectional area. $J_c$ is typically of order $10^8$A/cm$^2$, very large.
			% subsubsection critical_current_of_wire (end)


		% subsection screening (end)

			\subsection{The Intermediate State} % (fold)
			\label{sub:the_intermediate_state}
				Such effect depends on the \textbf{shape} of the sample.

				\subsubsection{In Strong Magnetic Fields} % (fold)
				\label{ssub:in_strong_magnetic_fields}
					
				% subsubsection in_strong_magnetic_fields (end)

				\subsubsection{Above Critical Current of a superconducting Wire} % (fold)
				\label{ssub:above_critical_current_of_a_superconducting_wire}
					
				% subsubsection above_critical_current_of_a_superconducting_wire (end)

			% subsection the_intermediate_state (end)

			\subsection{High-frequency Electrodynamics} % (fold)
			\label{sub:high_frequency_electrodynamics}
				\subsubsection{High-frequency Dissipation in Superconductors} % (fold)
				\label{ssub:high_frequency_dissipation_in_superconductors}
				
				% subsubsection high_frequency_dissipation_in_superconductors (end)
			% subsection high_frequency_electrodynamics (end)


	% section intro_to_electrodynamics_of_superconductivity (end)

	\section{The BCS Theory} % (fold)
	\label{sec:the_bcs_theory}
	
	% section the_bcs_theory (end)

\end{document}