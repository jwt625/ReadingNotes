%!TEX program = xelatex
%!TEX builder = latexmk
% or can be latexmk texify
%!TEX option = 
% -shell-escape -8bit % For minted package
%!TEX root = ...
\documentclass[8pt,a4paper,twocolumn]{article} % titlepage表示标题单独页
\usepackage[noindent]{ctex} % ctex套用英文标题格式 (建议在英文论文混排中文时使用) ,
% ctexcap套用中文格式 (等同于\documentclass{ctexart}) 
% \renewcommand{\figurename}{图}
% \renewcommand{\tablename}{表}
% \renewcommand{\contentsname}{目录}
% \renewcommand\refname{参考文献}
% \renewcommand{\thefigure}{\chinese{figure}} % 将图片计数改为汉字数字
% \renewcommand{\thetable}{\chinese{table}} % 将表格计数改为汉字数字
\usepackage[top=0.75in,bottom=0.75in,left=0.75in,right=0.75in]{geometry} % 页边距设置
% \usepackage{multicol}页面内多行包
\usepackage[CJKbookmarks]{hyperref} % 给pdf文档添加互动式链接和书签
\hypersetup{linktocpage=true}%make the link of the content on the number of page
% \userpackage{wrapfig} % 图文绕排
% \usepackage{xeCJK} % to get my Chinese name
% \setCJKmainfont{SimSun}
% \usepackage[parfill]{parskip} % 增加段间行距
\usepackage{amsmath,amssymb,esint} % 数学公式类宏包;最末为积分符号拓展
\allowdisplaybreaks[0]% 允许多行公式间换页, 用//*表示不允许换页
\numberwithin{equation}{section} % 公式编号包含章节
\usepackage{bm} % 加粗 (用于vector) 
\usepackage{mathrsfs} % mathscr font
% \usepackage{textcomp} % 符号包, 不能用于数学模式, 建议不要和SIunits混用
% \usepackage[squaren]{SIunits} % 科学单位包, 可以用于数学模式
% (为了统一不要和textcomp混用) , squaren选项消除和amssymb的冲突
\usepackage{siunitx} % 淘汰掉上面这个宏包吧, 现在用的是
% \num{123}, \si{kg.m/s^2}, 
% \si{\electronvolt\per\square\clight}, \SI{123}{\micro\metre}
\usepackage{extarrows} % 长箭头, 长等号etc.
\usepackage{graphicx} % 插图宏包
% \usepackage{picinpar} % 图文绕排
\usepackage{array} % 表格宏包
% \usepackage{longtable} % 长表格宏包
\usepackage{multirow} % 多行合并的表格宏包
% \usepackage{booktabs} % 表格线宏包
\usepackage{braket} % 狄拉克符号

% \usepackage[basic,box,gate,oldgate,ic,optics,physics]{circ} % 电路图宏包
% \usepackage[normalem]{ulem} % 下划线, 删除线等宏包, 参数表示不修改\emph{}格式
% \usepackage{mychemistry} % 化学宏包, 包含mhchem和chemfig
% \usepackage[version=3]{mhchem} % 化学宏包, 包含mhchem和chemfig
% \usepackage[symbol]{footmisc} % 脚注拓展, 选项表示用符号做脚注记号
% \usepackage{listings} % 代码段宏包
% \lstset{numbers=left,frame=shadowbox,%
% basicstyle=\ttfamily, commentstyle=\fontseries{lc}\selectfont\itshape, %
% columns=fullflexible, breaklines=true, escapeinside={(*@}{@*)}}
% \usepackage{minted} % 具有 Python 支持的代码宏包

% \renewcommand*{\vec}[1]{\bm{#1}} % 矢量的格式, 这里是加粗
\newcommand{\dif}{\,\mathrm d}
\newcommand\mi{\mathrm{i}}
\newcommand\e{\mathrm{e}} % 定义数学模式中常用的正体字符
\newcommand\Y{\mathrm{Y}}
\newcommand\cc{\mathrm{c.c.}}
\DeclareMathOperator{\Imag}{Im}
\bibliographystyle{unsrt}

\begin{document}\small
	\title{Principles of Lasers}
	\author{Wentao Jiang}
	\date{}
	\maketitle
	\tableofcontents

	\section{Concepts} % (fold)
	\label{sec:concepts}
		Spontaneous and stimulated emission, absorption

		Spontaneous (or radiative) emission:  
		\begin{align}
			&\nu_0 = (E_2-E_1)/h \\
			&\left( \frac{dN_2}{dt} \right)_{sp} = -AN_2 = -\frac{N_2}{\tau_{sp}}			
		\end{align}

		Non-radiative decay:
		\begin{align}
			&\left( \frac{dN_2}{dt} \right)_{nr}  = -\frac{N_2}{\tau_{nr}}
		\end{align}

		Stimulated emission \& absorption:
		\begin{align}
			&\left( \frac{dN_2}{dt} \right)_{st}  = -W_{21}N_2\\
			&W_{21}=\sigma_{21}F\\
			&\left( \frac{dN_1}{dt} \right)_{a}  = -W_{12}N_1\\
			&W_{12}=\sigma_{12}F
		\end{align}
		$\sigma_{21} $: stimulated emission cross section.\\
		$\sigma_{12} $: absorption cross section.

		Shown by Einstein:
		\begin{align}
			&g_2W_{21}=g_1W_{12}\\
			&g_2 \sigma_{21}=g_1 \sigma_{12}
		\end{align}

		Increase of photon flux $F$:
		\begin{equation}
			dF=\sigma_{21}F [N_2-(g_2N_1/g_1)]dz
		\end{equation}
		At thermal equilibrium:
		\begin{equation}
			\frac{N^e_2}{N^e_1}=\frac{g_2}{g_1}\exp\left( -\frac{E_2-E_1}{kT} \right)
		\end{equation}

		\subsection{The Laser} % (fold)
		\label{sub:the_laser}
			To generate laser, population inversion: $ N_2> g_2N_1/g_1 $\\
			Gain per pass: $\exp\left\{ \sigma [ N_2- g_2N_1/g_1]l \right\} $\\
			Critical inversion:
			\begin{equation}
				N_c = - [\ln R_1R_2 +2\ln (1-L_i)]/2 \sigma l
			\end{equation}
			$R_1, R_2$: power reflectivity of the two mirrors\\
			$ L_i $: internal loss per pass\\
			Define:
			\begin{align}
				& \gamma_1 = -\ln R_1 = -\ln (1-T_1)\\
				& \gamma_2 = -\ln R_2 = -\ln (1-T_2)\\
				& \gamma_i = -\ln(1-L_i)\\
				&\therefore N_c = \gamma/\sigma l\\
				&\gamma = \gamma_i + (\gamma_1+\gamma_2)/2
			\end{align}
		% subsection the_laser (end)
			
		\subsection{Pumping Schemes} % (fold)
		\label{sub:pumping_schemes}
			\begin{itemize}
				\item two-level: impossible to produce inversion
				\item three-level scheme
				\item four-level scheme: easier to realize inversion
				\item quasi-three-level: the ground level consists of sublevels
			\end{itemize}

			For most four-level and quasi-three-level lasers, the depletion of the ground level can be neglected, hence:
			\begin{equation}
				(dN_2/dt)_p=R_p
			\end{equation}
			$R_p$: pump rate. To achieve threshold, $R_p$ must reach $R_{cp} $.
		% subsection pumping_schemes (end)

		\subsection{Properties of Laser Beams} % (fold)
		\label{sub:properties_of_laser_beams}
			\begin{enumerate}
				\item Monochromaticity: much narrower (ten orders) than linewidth of spontaneous emission
				\item Coherence: temporal \& spatial coherence
				\item Directionality: $ \theta_d = \beta \lambda/D = \beta \lambda/ [S_c]^{1/2} $\\$S_c$: coherence area\\$\beta$: numerical coefficient\\$\lambda$: wavelength
				\item Brightness: $B = 4P/(\pi D \theta)^2$ for $\theta \ll 1 $\\with $\theta = \theta_d$, $B = \left( \frac{2}{\beta \pi \lambda} \right)^2P $
				\item Short time duration
			\end{enumerate}
		% subsection properties_of_laser_beams (end)

	% section concepts (end)

	\section{Interaction of Radiation with Atoms and Ions} % (fold)
	\label{sec:interaction_of_radiation_with_atoms_and_ions}
		
	% section interaction_of_radiation_with_atoms_and_ions (end)

	%\bibliography{CQED}
\end{document}